\chapter{Analog Measurement} \labchap{analog_measurement}
In electronics, there are two different types of signals: digital and analog.
Digital signals are either a "high" voltage, or a "low" voltage, representing either a logical 0 or logical 1.
These signals are great for threshold measurements such as "is there a cart present at the start of the ride, yes or no?", but cannot express a range of values.
Conversely, analog signals are better suited to expressing real-world values that naturally vary from a minimum to a maximum and can be any value in between.
Thus, analog signals are prevalent for devices that measuring "real" things such as speed, acceleration, rotation rates, barometric pressures, etc.

\begin{figure}[h!]
    \caption[Analog versus digital waveforms]{A digital square wave (top) versus an analog sinusoidal wave (bottom).
    Courtesy of \href{https://cdn.sparkfun.com/assets/learn_tutorials/2/1/5/analog_vs_digital.png}{SparkFun}}
    \labfig{analog_vs_digital}
    \centering
    \includegraphics[width=0.5\textwidth]{appendices/analog_measurement/analog_vs_digital.png}
\end{figure}

Most analog sensors are resistive types where their electrical resistance changes to scale with a real-world measurement range.
Ohm's Law defines the relation between voltage, ($V$) resistance ($R$), and current ($I$) as $V=IR$.
Therefore, if we can measure the voltage drop across a resistive sensor and correlate it with a manufacturer-provided equation or table, we can quantify a real-world phenomenon.
The resistance is best measured using a Wheatstone Bridge and an amplifier which can generate a voltage that is directly proportional to the change in resistance, $\delta R$

\section{Analog to Digital Conversion} \labsec{analog_to_digital_conversion}
Now that we can get a sensor reading as an analog (changing) voltage, we need to find a way to translate it into a useful form.
Decades ago, the analog voltage from a sensor was plotted onto a chart with respect to time and an analyst could run the conversion point by point.
Following the roller coaster example from before, park security is very unlikely to allow you to bring a full chart plotter on the coaster with you, and its even less likely to survive the trip or print a good readout.
We also want to work smarter and not harder, so we want a computer to perform the analysis for us. 
What can we do?
Since we live in the digital age, we can convert the analog signal to a digital one and then we can store it on a digital media device such as an SD card then load it directly into a computer program like Excel to analyze it.

This process is done with an analog to digital converter circuit or ADC. 
The ADC samples an analog waveform at a given frequency and places the voltage level into bins for each measurement.
There are $2^N$ bins in an ADC depending on its resolution, $N$.
For each measurement, the ADC will also encode the voltage reading as a binary number and save it to a register.

This encoded binary number is expressed as "counts" which can be read by a microcontroller or other digital system.
The number of counts recorded can be converted back to a voltage value at any time using the equation below.
However, this conversion will lose some of the original precision of the analog signal, depending on the resolution of the ADC.
A higher resolution will mean a more precise reading.

\begin{equation*}
    V = V_{cc} \frac{\text{counts}}{2^N}
\end{equation*}

\begin{figure}[h!]
    \caption[Analog to digital converter waveform]{A 3-bit ADC waveform converting an analog sinusoidal voltage wave (red) of period, $T$, to a digital representation (black).
    Each ADC bin is shown by a blue dashed line.}
    \labfig{adc_wave}
    \centering
    \includegraphics[height=2.5in]{appendices/analog_measurement/adc_wave.png}
\end{figure}

\begin{figure}
    \centering
    \begin{fitbox}[frametitle=Aside: Registers]
        Digital values are stored in memory as 1's and 0's.
        Each bit of memory can be stored in a contiguous piece called a ``register''.
        A digital system can access registers of memory to grab values and perform calculations on them.
        For example, an 8-bit ADC will store the encoding of an input signal in an 8-bit register.
        A microcontroller can then read this value from the ADC and convert it to a real-world value, or save it to non-volatile memory for logging.

        \includegraphics[width=\textwidth]{appendices/analog_measurement/gated_latch.png}
        \caption{A gated latch circuit used to store 1 bit of memory.}
    \end{fitbox}
\end{figure}