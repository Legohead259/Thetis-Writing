\chapter{Gyroscope Coriolis Effect Proof} \labchap{gyroscope_proof}

The position of the center of mass in the gyroscope body  frame is given by:

\begin{equation}
    B_c = 
    \begin{bmatrix}
        x \\
        y 
    \end{bmatrix}
\end{equation}

The inertial velocity of the mass is given by the positional derivative and the tangential velocity due to rotation:

\begin{equation}
    \dot{B_c} = 
    \begin{bmatrix}
        \dot{x} \\ 
        \dot{y}
    \end{bmatrix}
    + B_\omega \times B_r =
    \begin{bmatrix}
        \dot{x} - \omega y \\
        \dot{y} + \omega x
    \end{bmatrix}
\end{equation}

The inertial acceleration is the next derivative given by:

\begin{equation}
    \ddot{B_c} = 
    \begin{bmatrix}
        \ddot{x} \\
        \ddot{y}
    \end{bmatrix} = 
    \begin{bmatrix}
        \dot{x}-\omega y \\
        \dot{y} + \omega x
    \end{bmatrix}
    + B_\omega \times B_{\dot{r}} = 
    \begin{bmatrix}
        \ddot{x} - 2\omega\dot{y} - \omega^2 x \\
        \ddot{y} + 2\omega\dot{x} - \omega^2 y
    \end{bmatrix}
\end{equation}

The first element in $(3)$ represents the acceleration experienced by the x-axis (driven axis). This axis is actively controlled by the gyroscope driving circuit. The second element is that of the sensing axis (y-axis). Recall Newton’s Second Law of Motion, $F=ma$:

\begin{equation}
    F_y=mB_{\ddot{c}, y} = m(\ddot{y}+2\omega\dot{x}-\omega^2y)
\end{equation}

If the mass starts at the resting position, $y=\dot{y}=\ddot{y}=0$. Therefore, we get:

\begin{equation}
    F_y=2m\omega\dot{x}
\end{equation}

Since the mass is displaced along the x-axis at a high frequency and short distance, $\dot{x}$ is significant and the Coriolis effect generates a high amplitude, proportional displacement in the y-axis. The tuning fork configuration essentially doubles the displacement and makes capacitive detection easier. Additionally, when undergoing linear acceleration, the masses move equally, minimizing sensitivity to shock, vibration, and tilt.