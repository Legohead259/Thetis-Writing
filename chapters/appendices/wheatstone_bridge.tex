\chapter{Wheatstone Bridge} \labchap{wheatstone_bridge}
The Wheatstone Bridge is a configuration of four resistors in two voltage dividers as shown in Figure \ref{fig:wheatstone_bridge} below that is sensitive to minute changes in resistance.
Resistors $R_1$ and $R_2$ form the first divider and $R_3$ and $R_4$ form the second divider.
The bridge has at least three different solutions, depending on the configuration and desired application.
For applications where $R_1$, $R_2$, and $R_3$ are known to a high precision, then $R_4$ can be determined to an equally high precision by adjusting $R_3$ until the voltage potential between points $B$ and $D$ is near 0, i.e. the bridge is "balanced".
However, for most embedded applications, like those described in further sections, $R_4$ cannot be determined by balancing the bridge, so either the general solution or linearization must be used.
For simplicity, only the linearization solution will be examined here, the full derivation and explanation for all three solutions can be found in Appendix \ref{chap:voltage_divider}.

\begin{figure}[h!]
    \caption{A generalized wheatstone bridge}
    \labfig{wheatstone_bridge}
    \centering
    \includegraphics[height=2.5in]{appendices/wheatstone_bridge/wheatstone_bridge.png}
\end{figure}

For the linear solution, the circuit adds an operational amplifier between points $D$ and $B$ and assumes that $R_1+R_2=R_3=R_0$ and $R_4 = R_0 + \Delta R$ and the op-amp is an ideal component ($R_{\text{amp}}$ = 0).
Because of this, the voltage potential at Point $B$ will have a constant value of:

\begin{equation*}
    V_B = \frac{R_0 V_s}{R_0 + R_0} = \frac{V_s}{2}
\end{equation*}

\begin{figure}[h!]
    \caption{A linearized wheatstone bridge}
    \labfig{wheatstone_bridge_linearized}
    \centering
    \includegraphics[height=2.5in]{appendices/wheatstone_bridge/wheatstone_bridge_linearized.png}
\end{figure}

Likewise, the op-amp will force the voltage at Point $D$ to have the same voltage as $B$ such that, $V_D = V_B = \frac{V_s}{2} $.
This forces a constant current of $\frac{V_s}{2R_0} $ through $R_3$ and into the sensor.
By Ohm's law, the voltage across the sensor will therefore be:

\begin{align*}
    V_4 &= \frac{V_s}{2R_0} \cdot R_0(1+\Delta R) \\
        &= \frac{V_s}{2} + \frac{V_s}{2} \Delta R
\end{align*}

By applying Kirchoff's voltage law, the potential between the amplifier output and ground, $V_{out} $, is:

\begin{align*}
    V_{out} &= -V_4 + V_D \\
            &= -\left( \frac{V_s}{2} + \frac{V_s}{2} \Delta R\right) + V_D \\
    V_{out} &= -\frac{V_s}{2} \Delta R
\end{align*}

The measured output voltage now linearly changes with the resistive sensor irregardless of if the sensor changes linearly itself.
If the manufacturer of the sensor provides a table or equation that relates the sensor resistance to a real-world value, we can find the sensor resistance via:

\begin{align*}
    R_4 &= R_0 + \Delta R \\
        &= R_0 - \frac{2V_{out}}{V_s}
\end{align*}