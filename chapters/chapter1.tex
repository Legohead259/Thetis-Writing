%%-------------------------------------Chapter 1------------------------------------------
\chapter{Introduction}\label{chap:intro}
We all view the world through different lenses.
Biologists look at plants and animals and seeks to understand their interactions;
similarly, a scientist considers the universe around us and ponders what drives and constrains it;
a psychologist examines human behavior and analyzes its origin and meaning.
Humanity has always tried to study the way things work and figure out why they happen.
That is the natural product of our curiosity and one of the many things that distinguishes us from other animals.

The fundamental problem that is considered throughout this thesis is sensing the inertial characteristics of a body and determining its orientation.
The solutions to this problem are not new.
It has been well-studied, well-documented, thoroughly proven out, and is even available to the masses at every moment of every day - even though they may not be aware they are using it.
The purpose of the content herewithin is not to unveil something new or revolutionary.
Rather it is the culmination of years of viewing this problem through the engineer's lense and pondering the questions, "how does this work?" and "how can I make it work for me?"
This is a guide into how an engineer can look at a problem and tinker with it and research it and develop something for it that enhances their understanding.
While the final product may not be itself an innovation, the skills and knowledge required to get there can always lead to greater things.