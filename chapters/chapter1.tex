%%-------------------------------------Chapter 1------------------------------------------
\chapter{Introduction}\label{chap:intro}
How do we sense the world around us?
What compels us to understand our surroundings and categorize what we are feeling and contextualize the sensations based off our experience and knowledge?
Survival.
Survival is the driving force that has honed the human machine into the empire-building globe spanning species it is today.
Without the ability to sense the environment around us and have the drive to survive, we would have never become what we are, nor achieve any of our potential.

The development of robotic systems over the last five decades has matched that of evolution.
Much like the first microbes, these robots spent their useful lives serving one purpose or another, never anything too complicated, and rarely anything consecutively.
Gradually as our technology and our understanding advanced, we could afford to add complexity to the next generation of robots;
We began to develop the math that added additional degrees of movement.
We began to shrink the electronics which allowed us to add more of them.
We began increase the efficiency of the parts, which allowed us to use different power supplies - like batteries.
We began to develop abstract ways of explaining to robots how to move, how to think, how to process information that allow them to do incredible things .
Modern robotics have far outstripped their comparatively ancient predecessors and are beginning to be present in the average person's day-to-day life.

In the marine field, robotics are becoming more prevalent as ship systems are becoming more automated.
We are approaching the time where ships can be fully unmanned and operated in either a remote-controlled or autonomous mode.
Autonomy packages even have the capability to "swarm" together to perform a common mission, or automatically replace vessels in the formation that suffer casualties [CITE].
However, one thing hampers these unmanned vessels during their operations: sensing their environment and conditions.

An experienced human captain is able to feel how their ship reacts to the water and air around them. 
The rocking and rolling of a ship is a normal experience for a sea-farer and irregularities can be felt and reacted to.
For instance, if a small flat-bottom hull is approaching noticeable waves, the coxswain may choose to slow the boat down so that the impacts are reduced.
This makes getting over those incoming waves gentler on the crew as well as the craft itself; we have a baked-in survival instinct that tells us that hitting a wave hard is not rewarding for our health prospects and that we should minimize and avoid it, if at all possible.
Computers, on the other hand, do not share this survival instinct.
They are tasked with a mission directive and often the reward or utility of not beating the vessel up on the waves is not considered.

In this thesis, we explore how can begin to implement survival instinct into unmanned surface vessels by integrating a novel sensing and data logging package as well as modeling the vessel response to waves, in real-time using machine learning processes.
This all culminates in a safe navigation (SafeNAV) algorithm that informs a central controller about the maximum speed limit it can achieve in its current wave environment.
