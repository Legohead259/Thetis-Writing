%%--------------------Chapter 3------------------------
\chapter{Thetis Design} \label{chap:thetis_design}

\section{Stakeholders} \label{sec:stakeholders}
As with any project, stakeholders are a crucial part of the design process.
Stakeholders drive certain requirements that the device must achieve in order to be accepted for operation.
On an individual level, the committee overseeing this thesis are the major stakeholders as they have a vested interest in the success or failure of the project.
However, there are some organizational-level stakeholders that have also expressed interest in the project and provided some feedback for their requirements.
A summary of the stakeholders is provided in Table \ref{tab:stakeholders}.

\begin{table}
	\caption{A summary of stakeholders of the Thetis device}
	\label{tab:stakeholders}
	\centering
	\begin{tabular}{|p{0.3\linewidth} | p{0.6\linewidth}|}
		\hline
		\rowcolor[gray]{0.8}
		\multicolumn{1}{|c|}{\textbf{Stakeholder}} & \multicolumn{1}{|c|}{\textbf{Description}} \\
		\hline
		Committee members & Individual professors who have expressed interest in the project and have agreed to assist in its development. Specifically, Dr. Wood and Dr. Weaver would like to deploy Thetis on university projects; Dr. Gutierrez has provided many requirements on the performance of the instrumentation; and Dr. Silaghi is interested in its application to autonomous navigation. \\
		\hline
		Florida Institute of \newline Technology & The university has several classes and projects where the Thetis design could be useful. The Instrumentation Design and Analysis class used Thetis as a demonstrator for designing PCBs and field experiments. Thetis was also designed with Surf Engineering Analysis in mind for students to have a new open source sensor to experiment with. \\
		\hline
		Maritime Tactical \newline Systems (MARTAC) & This corporation has expressed interest in using Thetis for research and characterizing the performance of their high speed catamarans in open-water testing \\
		\hline
		NSWC Carderock - \newline Combatant Craft Division (CCD) & This three-letter agency of the government has expressed interest in using Thetis as a testing and evaluation tool for small unmanned crafts \\
		\hline
	\end{tabular}
\end{table}

\subsection{Hands-On Users} \label{ssec:hands_on_users}

\section{Design Rationale} \label{sec:design_rationale}
Thetis is envisioned as an open-source all-in-one data logging solution for use in research projects.
The device incorporates multiple sensors, GPS tracking, and a WiFi-capable microcontroller in order to enable as many features as can be envisioned by the end users.
One of the driving considerations was the small footprint.
Thetis Revision F is designed to fit into one of the smallest IP67-rated enclosures available on the market [LINK].  
This tiny form factor allows it to be inconspicuously mounted to any floating body like surfboards, scale models, or wave buoys without upsetting their inertial characteristics or impeding nominal operation.

Further discussions with stakeholders later in the design process envisioned a new iteration, Revision G, that had a larger footprint, but added more capabilities like CANbus integration and several connection ports for different communication protocols.
This revision is meant for deployment on vessels that have a NMEA2000 communications bus.
This bus allows the device to be powered from the boat's power supply and communicate data to a central controller.
These features are extremely important for the final application where Thetis is meant to feed data into a navigation algorithm for the safe passage of unmanned vessels.

\subsection{Problem Description} \label{ssec:problem_desc}
Tracking the inertial movements of small floating bodies in-situ is difficult for small-scale or student-led experiments.
The price of the measurement instruments and the data acquisition computers (DAQs) offers a high bar for entry for these projects.
Also, the size of these instruments and DAQs can negatively impact the performance or operation of small floating bodies so they cannot be effectively used.
This forces classrooms and organizations to either neglect collecting inertial data or using bulky, unreliable prototype setups using off-the-shelf components

\subsection{Mission Statement} \label{ssec:mission_statement}
Thetis aims to democratize the inertial measurement and tracking space for small scale experiments by implementing an open-source, feature-rich, all-in-one solution to monitoring the movements of floating bodies.

\subsection{Stakeholder Requirements} \label{ssec:stakeholder_reqs}
Interviews with the stakeholders occurred over several months and informed a set of requirements that they determined were necessary for the project's success.


\subsection{Feasibility and Risk Identification} \label{ssec:feasibility_risk}
The following tables provide the supporting documentation to the requirement feasibility assessment for technical, cost and schedule, organizational, and political and operational. The conclusion is that all requirements have been proven feasible with current technology.

\begin{landscape}
% ==================================
% === STAKEHOLDER REQUIREMENT XX ===
% ==================================

% \paragraph{S.R. XX} - The requirement

% {\fontsize{10pt}{11pt}\selectfont
% \begin{longtable}{| p{0.12\linewidth} | p{0.16\linewidth} |  p{0.20\linewidth} | p{0.08\linewidth} | p{0.20\linewidth} | p{0.08\linewidth} |}
%     \hline \endlastfoot
	
%     \hline
%     \rowcolor[gray]{0.8}
%     \multicolumn{6}{|c|}{ } \\
%     \hline
%     \textbf{Stakeholder:} & \multicolumn{5}{|l|}{Wilford Erasmus, POC of the three-letter agency} \\
%     \hline
%     \textbf{Rationale:} & \multicolumn{5}{|L{0.8\linewidth}|}{s} \\
%     \hline
%     \textbf{Fit Criterion:} & \multicolumn{5}{|L{0.8\linewidth}|}{} \\
%     \hline
%     \rowcolor[gray]{0.8}
%     \multicolumn{6}{|c|}{ } \\
%     \hline
%     \textbf{Risk} & \textbf{Risk Issue} & \textbf{Risk Consequence} & \textbf{Initial Risk} & \textbf{Risk Mitigation} & \textbf{Risk \newline After Mitigation} \\
%     \hline
%     Technical \newline Assessment &  &  & \cellcolor{} &  & \cellcolor{} \\
%     \hline
%     Cost and Schedule \newline Assessment &  &  & \cellcolor{}  & & \cellcolor{}  \\
%     \hline
%     Organizational assessment &  &  & \cellcolor{}  &  & \cellcolor{}  \\
%     \hline
%     Political and Operational Assessment &  &  & \cellcolor{}  &  & \cellcolor{} 
%     \label{tab:srXX_feasibility}
% \end{longtable}
% }

% \newpage

% ==================================
% === STAKEHOLDER REQUIREMENT 01 ===
% ==================================

\textbf{SR 01} - The system shall be able to record acceleration, rotation rate, orientation, and position

{\fontsize{8pt}{8pt}\selectfont
\begin{longtable}{| p{0.12\linewidth} | p{0.16\linewidth} |  p{0.20\linewidth} | p{0.08\linewidth} | p{0.20\linewidth} | p{0.08\linewidth} |}
	\hline \endlastfoot
	
	\hline
	\rowcolor[gray]{0.8}
	\multicolumn{6}{|c|}{ } \\
	\hline
	\textbf{Stakeholder:} & \multicolumn{5}{|l|}{Dr. Stephen Wood} \\
	\hline
	\textbf{Rationale:} & \multicolumn{5}{|l|}{The system needs to be able to record the inertial characteristics of a floating body} \\
	\hline
	\textbf{Fit Criterion:} & \multicolumn{5}{|p{0.8\linewidth}|}{This will be accomplished using a 9-DOF IMU and GPS receiver with accuracies not to exceed one standard deviation of a reference source} \\
	\hline
	\rowcolor[gray]{0.8}
	\multicolumn{6}{|c|}{ } \\
	\hline
	\textbf{Risk} & \textbf{Risk Issue} & \textbf{Risk Consequence} & \textbf{Initial Risk} & \textbf{Risk Mitigation} & \textbf{Risk \newline After \newline Mitigation} \\
	\hline
	Technical \newline Assessment & The IMU and/or GPS will report measurements that have a high margin of error and little consistency & Worthless data for analysis & \cellcolor{yellow} Medium & Selection of sensors that have decent accuracy and low drift. \newline Use a \emph{tuned} Kalman filter to improve reported sensor accuracy & \cellcolor{green} Low \\
	\hline
	Cost \newline Assessment & Chip shortage as a result of the COVID-19 pandemic. & Unable to find appropriate components. \newline Any found components are prohibitively expensive & \cellcolor{yellow} Medium & Find components that are in stock and order in bulk. & \cellcolor{yellow} Medium \\
	\hline
	Schedule \newline Assessment & Sensor fusion algorithms are difficult to implement and tune. & Schedule overrun trying to tune the algorithms. \newline Inaccuracies introduced through improper tuning. & \cellcolor{yellow} Medium & Good programming practices to make tuning easier during testing. & \cellcolor{green} Low \\
	\hline
	Organizational Assessment & Lack of subject matter experts & Project cost and schedule delays & \cellcolor{green} Low & In-house team available. \newline Can reduce scope, as needed. \newline Proper documentation of progress and scheduled design reviews and consultations & \cellcolor{green} Low \\
	\hline
	Operational Assessment & Unreliable sensors. & Device fails and does not recover during testing; lost data & \cellcolor{yellow} Medium & Reliability analysis and testing required. & \cellcolor{green} Low
	\label{tab:sr01_feasibility}
\end{longtable}
}

\end{landscape}

\subsection{Quality Functional Deployment} \label{ssec:qfd}

\section{Concept of Operations} \label{sec:conops}

\section{Conceptual Design} \label{sec:conceptual_design}

\subsection{Functional Block Diagram} \label{ssec:block_diagram}

\subsection{Functional Flow Diagram} \label{ssec:flow_diagram}

\subsection{Available Commercial Off the Shelf Products} \label{ssec:cots_products}

\subsection{Analytical Hierarchy Process} \label{ssec:ahp}

\subsubsection*{Inertial Measurement Unit} \label{sssec:ahp_imu}

\subsection{Failure Mode, Effect, and Criticality Analysis} \label{ssec:fmeca}