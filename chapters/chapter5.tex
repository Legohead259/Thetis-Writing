%%-----------------Chapter 5---------------------------
\chapter{Verification and Validation}
Now that we have an instrument designed, how can we determine if it meets requirements or not?
The system can be verified and then validated using a variety of techniques.
First, to verify that a requirement is present, the component or subsystem can be examined independently.
If it passes examination, we can update the traceability matrix to reflect that the requirement is now verified.
Once the feature is integrated with the entire system, then we can determine if it is validated or not.
Instead of examining the feature independently, it will be done so with the entire system to ensure that it functions properly.
These examinations can fall under four main types:

{
\renewcommand{\descriptionlabel}[1]{\hspace{\labelsep}\textbf{#1}}
\begin{description}
    \item[Inspection:] The nondestructive examination of a product or system using one or more of the five senses (visual, auditory, olfactory, tactile, taste). 
    It may also include simple physical manipulation and measurements.
    
    \item[Demonstration:] The manipulation of the product or system as it is intended to be used to verify that the results are as planned or expected.								
    
    \item[Test:] The verification of a product or system using a controlled and predefined series of inputs, data, or stimuli to ensure that the product or system will produce a very specific and predefined output as specified by the requirements.								
    
    \item[Analysis:] The verification of a product or system using models, calculations and testing equipment.
    Analysis allows someone to make predictive statements about the typical performance of a product or system based on the confirmed test results of a sample set or by combining the outcome of individual tests to conclude something new about the product or system.
    It is often used to predict the breaking point or failure of a product or system by using nondestructive tests to extrapolate the failure point.
\end{description}
}

For the following chapter, we shall examine each of Thetis' design capabilities from Tables \ref{tab:threshold_capabilities} through \ref{tab:stretch_capabilities} and verify their presence on the latest design.
Then, we shall do the same for the stakeholder requirements.

\section{Threshold Capabilities}

\paragraph*{101 - The system must be housed in an IP67-rated enclosure} This capability was initially verified by locating a suitable enclosure for Thetis and confirming with the manufacturer through the product data sheet that the system was properly rated.
Then, it was validated by placing the enclosure onto a Remotely Operated Vehicle (ROV) and descending down to 20-feet of depth while the ROV performed a mission.
Paper towels were placed within the enclosure so that if any water breached the seals, it would be immediately apparent.
This represented a worst-case scenario and the enclosure held its seal meaning that it exceeded the capability requirement.

[TODO: Insert picture of Thetis on ROV]

To represent a more realistic use case, Thetis was placed into a cutout on a surfboard and deployed into the ocean to catch some waves.
Out nine separate deployments, the case seal failed three different times:

\begin{enumerate}
    \item A screw was not present in one corner, thereby not clamping that section of the sealing material, 
    \item a screw was over-torqued and caused the area around the screw to crack, bypassing the seal, and 
    \item the same case as before was used accidentally.
\end{enumerate}

All three failures were due to operator error proving that careful procedures need to be implemented for actual deployments.
However, when the case was properly sealed by operators, the seals held and the electronics within were not damaged, even when completely submerged and subjected to dynamic forces as the surfboard rolled and slammed into waves.

[TODO: Insert picture of Thetis on surfboard]

\paragraph*{102 - The system enclosure must fit within a volume of 8" x 5" x 1.25"} Like Capability 101, this was initially verified by inspection during the search for this enclosure and confirming the enclosure dimensions with the manufacturer.
Then, it was validated by placing it securely into the cutouts in the surfboard made for the iPhone 6S - the device Thetis is intended to replace, as shown in Figure \ref{fig:thetis_surfboard}.

\paragraph*{103 - The system software and firmware will be fully open-source} This capability is both verified and validated by inspection as the code is readily accessible on GitHub.
The firmware is broken into several sub repositories: \href{https://github.com/Legohead259/Thetis-Firmware}{Thetis-Firmware}, \href{https://github.com/Legohead259/ThetisLib}{ThetisLib}, \href{https://github.com/Legohead259/xioAPI-Arduino}{xioAPI-Arduino}, \href{https://github.com/Legohead259/Timer-Events-Arduino}{Timer-Events-Arduino}, and \href{https://github.com/Legohead259/Fusion-Arduino}{Fusion-Arduino} all of which are under the MIT license and available.
Thetis has several tangential software packages that are also open source such as the \href{https://github.com/Legohead259/Thetis-Scripts}{scripts repository} used for data processing and analysis, the \href{https://github.com/xioTechnologies/x-IMU3-Software}{x-IMU3 GUI} used to visualize data and log from a host computer, and the \href{https://github.com/Legohead259/Thetis-Calibration}{code for the calibration machine}.

\paragraph*{104 - The system shall record inertial measurements at a minimum frequency of 64 Hz} This capability requires a demonstration in order to be verified and validated.
We can initially inspect the firmware to ensure that the inertial measurements are taken every 15.6 milliseconds, but it is not guaranteed that the system can consistently take measurements at that speed.
Therefore, the best way to demonstrate this capability was during the calibration procedure detailed in Section \ref{sec:calibration_methodologies}.
Thetis was set to record at 64 Hz and when the data was offloaded, it was confirmed to be taken at the appropriate interval.
Therefore, this capability has been verified and validated within the system.

\paragraph*{105 - The system shall be able to data locally for up to 4 hours continuously} We can analyze the size of logging messages and SD card to determine if this capability is verified.
In the latest version of the firmware, five messages are published to the data storage device: position, inertial, magnetic, quaternion, and euler angle.
Combined, these messages take 198 bytes of space and occur, on average, 64 times per second for 12,672 bytes per second.
There are 14,400 seconds in 4 hours, so multiplying these values together, we get 182.5 megabytes of storage required for 4 hours of use.
The microSD cards used throughout testing are at least 4 gigabytes which gives an estimated 88 hours of continuous logging.
This requirement was validated by running Thetis for four hours continuously and verifying that the log file was successfully created and maintained for that period.

\begin{equation} \label{eq:storage_time}
    t_{\text{samples}} = \frac{N_{\text{storage}} [\text{Bytes}]}{198 [\text{Bytes}] \times 64 [\text{s}^{-1}] \times 3600 \left[\frac{\text{s}}{\text{h}}\right]} = \frac{4 [\text{GB}]}{198 [\text{Bytes}] \times 64 [\text{s}^{-1}] \times 3600 \left[\frac{\text{s}}{\text{h}}\right]} = ~88 [\text{h}]
\end{equation}

\paragraph*{106 - The system shall be able to operate for up to 4 hours continuously} For this capability, we can verify it by analysis like the previous one.
We need to start with a couple of assumptions:

\begin{enumerate}
    \item Battery capacity is 420 mAh with 3.7V nominal voltage,
    \item Current consumption without WiFi enabled is ~50 mA, and
    \item Current consumption with WiFi enabled is ~120 mA while transmitting
\end{enumerate}

The latter two assumptions are based on a zeroth-order estimate by summing together the estimated current consumption of the various components as listed in their data sheets.
We can then make a zeroth-order estimate using Equation \ref{eq:battery_life}.
This yields an estimated continuous battery life of 9.4 hours without WiFi and 3.9 hours with WiFi.
An important note about the WiFi estimate is the assumption that it is constantly transmitting.
In reality, this may not be accurate so the battery life may be longer.
If it is below the four-hour threshold, then certain mitigations can be implemented like a burst-mode transmission of data every couple of second or minutes.

\begin{equation} \label{eq:battery_life}
    t_{\text{battery}} = \frac{3.7 [\text{v}] \times 420 [\text{mAh}]}{3.3 [\text{V}] \times I_{\text{mode}} [\text{mA}]}
\end{equation}

This capability was validated by running Thetis for four hours continuously from full battery power and ensuring that the battery voltage at the end of the test was within the safe operating limits.
Specific power consumption tests were not performed due to the technical complexity of precisely measuring current draw.

\paragraph*{107 - The system shall have a simple human interface mechanism for status and logging} This is another capability that is straight forward to verify and validate using inspection techniques.
First, to enable logging, an operator only needs to hold the ``log'' button for a half second and the same to stop logging.
To convey status, the RGB LED on-board changes color and pattern.
By referencing the current RGB LED color and pattern with the diagnostic LED table, then the operator can know what the system is doing.
These features were used extensively throughout testing with great success.

\paragraph*{108 - The firmware shall be open architecture} This capability is challenging to fully define and implement, hence its relatively low priority in the threshold category.
To verify this capability has been met, we should consider the difficulty of adding a new feature or component to the firmware.
The firmware uses an object-oriented approach with a star topology as discussed in Section \ref{sec:firmware}.
This means that a new feature can be added by putting it into the appropriate class and tying it to other classes/functions through the main \lstinline[style=customInline]|Thetis| object.
Similarly, we can add a new component by creating a new class in the library and then implementing it in the main class.

Validating this capability will require more research out the scope of this thesis to ensure a proper software architecture is implemented and followed.

\paragraph*{109 - The system will be fully documented} This is another capability that is easy to validate and verify through testing.
Multiple groups of students and stakeholders will be asked to perform simple tasks using Thetis such as replacing a component, assembling the board, adding a firmware feature, or performing calibration.
If the participants are able to perform the task using the available documentation, then this capability has been verified and validated.
Otherwise, the documentation needs to created or modified accordingly.

\paragraph*{110 - The system shall use version control software to track changes} As discussed in Section \ref{ssec:version_control}, GitHub is a centralized VCS solution that enables version history and tracking.
Since all of the software and firmware for Thetis is on GitHub, this capability has been thoroughly verified and validated.
Similarly, all of the hardware was designed in Fusion 360 which implements its own VCS solution and then backed up to GitHub.

\paragraph*{111 - The operator shall be }