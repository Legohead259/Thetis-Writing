%%-----------------Chapter 6---------------------------
\chapter{Future Efforts and Potential Applications}
As you may have noticed, this thesis was an intensive, multi-disciplinary effort that required in-depth knowledge of electronics, board design, software engineering, and systems engineering.
Because of the breadth and depth required by this thesis, some areas were not covered due to technical or temporal constraints.
A large portion of the software was only finalized in the months leading up to finishing this thesis and a hardware fault cost another couple of months of development work.
Therefore, there is a lot left unfinished that is intended for future students to pick up as their own research projects.
Some of these efforts are detailed in this chapter.

\section{Calibration}
As discussed at the end of Chapter \ref{chap:calibration}, the calibration procedure failed for the inertial sensors.
The specific reasons could not be determined before the thesis needed to be completed and were out of the technical scope anyways.
Therefore, it will be required in the future to examine these procedures and determine the source of the flaws.

In order to accomplish this, the objective function and constraints should be examined first.
It is likely that the objective function was not used properly with the optimization procedure, producing results that fell into a local minima that satisfied the problem, but did not satisfy the calibration application.
Therefore, the optimization function should be modified to include additional constraints or boundaries to limit the range of ``acceptable'' results from the objective function.
This may improve the accuracy of the calculated misalignment matrices and sensitivity vectors.

Additionally, these procedures should be more thoroughly tested and evaluated across a range of boards and types of sensors.
The calibration script created for this thesis is not sufficient for this task, so the concepts employed by it should be expanded as required.
The measurements can be compared between the boards before and after calibration to determine the efficacy of the proposed procedures.

\section{Hardware Revision F6}
Based on testing and interviews with Dr. Madgwick [CITe - madgwick interviews], another hardware revision, Revision F6, was created to address some of the concerns with Revision F5.
First, the secondary flash storage chip, the XTSD, was removed entirely since it was redundant and caused the hardware issues that crippled development on Revision F5 for months.

The space created by removing this chip allowed the MARG array to moved to a section of the board that could be mechanically isolated from the rest of the board.
This was necessary because MEMS sensors are affected by strain and in the configuration on Revision F5, the sensors were located at a point of max strain when the board was screwed into the enclosure.
This would affect precision and could cause measurements to slightly vary depending on the torque of the screws used in the assembly.

Additionally, the \lstinline[style=customInline]|data ready| interrupts from the sensors were attached to the microcontroller which should improve performance by switching the measurement method to an interrupt-based one versus polling.
By switching to an interrupt-based measurement method the readings can be taken at a much higher sample rate and be more accurate to the recorded timestamp.
This method is also less computationally intensive on the microcontroller.

Then the microcontroller was changed to one that had an antenna attached directly to its PCB, removing the need for an external antenna.
On the ESP32-S2 microcontroller, the chip would refuse to boot when an external antenna was attached.
This issue was never fully investigated, but the onboard antenna should negate this problem and simplify the overall assembly.

Finally, the power supply was substituted for a more efficient DC-DC converter which should improve efficiency and extend battery life.

This revision has been designed in ECAD, but was not ordered or built due to time constraints.
Therefore, in the future, a student or group of students can build and verify and validate this design using the procedures laid out for Revision F5.

[TODO: Insert picture of Revision F6 and schematic]

\section{Software Improvements}
There are several things that are feasible with the hardware available that were not able to be implemented before the thesis needed to be completed.
Most of these features are not explicitly required by the system requirements, but they could improve the quality and usability of Thetis.

\subsection{Wireless}

\subsection{Settings}

\subsection{Data Storage}


\subsection{Sensors}
Many of the sensors implemented in the firmware are based upon the Adafruit libraries for those components.
While this simplifies the integration, it bloats the memory and storage requirements.
This also creates a dependency on third-party software which could be problematic in some situations.
Therefore, it would be nicer to have all of the sensors packaged individually in libraries that are self-contained and conform to the I2CDevLib [TODO: add footnote to I2CDevLib] standards.
This would improve the code performance and reduce memory requirements.

Additionally, the \lstinline[style=custominline]|Fusion| library has implemented some new features for accelerometer and magnetometer rejection and status flags that would be useful for the entire sensor fusion process.
Therefore, the firmware should be migrated to this new version and have these features implemented.

\section{More Verification and Validation}

\subsection{Wireless}

\subsection{Power}

\section{Calibration Machine}

\subsection{Requirements}

\subsection{Block Diagram}

\subsection{Concept of Operations}

\section{Floating Body Testing}

\subsection{Surfboard}

\subsection{Wave Buoy}

\subsection{Model Barge}

\section{WEAVE}