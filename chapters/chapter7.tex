\chapter{Conclusion} \label{chap:conclusion}
A low-cost, low-profile inertial data logger was designed, tested, and validated throughout this thesis.
I set out to design the board as a solution that students could use in their projects in and out of the classroom and as a replacement for some of the current equipment.
In putting the board together, I did my best to learn and follow system engineering best practices for a small embedded device like this and thoroughly validate the design before delivering it.
There were some problems, a couple of hiccups, and one or two or three major disasters along the way, but ultimately Thetis works.
It is an all-in-one datalogging solution, that is beneath \$200 in cost, is open source and publicly available, and is capable of recording inertial characteristics in 9 degrees of freedom with a GPS radio providing positional fix.

By reading through this thesis we have considered the two research questions posed in the introduction.
We have learned how inertial measurement units work, and we have learned how we can design our own instrumentation board to capture and report those readings.
With the capabilities found in Thetis, we can now move on to larger experiments incorporating it and continue to innovate in the research field.
All the while remembering to teach the next generation everything we know and providing them a platform to exceed us.